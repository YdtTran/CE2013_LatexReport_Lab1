\subsection{Exercise 3}
Given the following circuit, students rearrange the circuit to clarify its serial and/or parallel topology. Next, apply the knowledge you’ve learned to find the equivalent resistance value between two circuit terminals A and F, the voltage values at A, B, C, D, and E. Finally, perform the simulation to check your calculation.
\begin{figure}[H]
    \centering
    \includegraphics[width=0.6\linewidth]{graphics/lab1_ex3.jpg}
    \caption{Find the whole-circuit equivalent resistance and the voltages at A, B, C, D, and E}%
    \label{fig:6_3_circuit}
\end{figure}

\subsubsection{Rearrange the circuit}

By drawing a wire with current source I1, A, B, C, D, and E, we can clarify the circuit topology. As follows:
\begin{figure}[H]
    \centering
    \includegraphics[width=0.6\linewidth]{graphics/lab1_ex3_rear.jpg}
    \caption{Rearranged circuit}%
    \label{fig:6_3_rearranged_circuit}
\end{figure}
\subsubsection{Calculation}
As the rearranged circuit showed in Figure \ref{fig:6_3_rearranged_circuit}, we can calculate the equivalent resistance $R_{AF}$ by the following steps:
First, we calculate $R_{DE}$. Because $R9 \parallel (R4+R5)$, we have:
\vspace{-0.15cm}
\[
    R_{DE}=\dfrac{1}{\dfrac{1}{R_9} + \dfrac{1}{R_4 + R_5}}=\dfrac{1}{\dfrac{1}{6} + \dfrac{1}{1+2}}=2(k\Omega)
\]
Next, we calculate $R_{CE}$. Because $R6 \parallel (R3 + R_{DE})$, we have:
\vspace{-0.15cm}
\[
    R_{CE}=\dfrac{1}{\dfrac{1}{R_6} + \dfrac{1}{R_3 + R_{DE}}}=\dfrac{1}{\dfrac{1}{6} + \dfrac{1}{10 + 2}}=4(k\Omega)
\]
Now, we calculate $R_{BE}$. Because $R10 \parallel (R2 + R_{CE})$, we have:
\vspace{-0.15cm}
\[
    R_{BE}=\dfrac{1}{\dfrac{1}{R_{10}} + \dfrac{1}{R_2 + R_{CE}}}=\dfrac{1}{\dfrac{1}{6} + \dfrac{1}{2+4}}=3(k\Omega)
\]
We then calculate $R_{BF}$. Because $R8 \parallel (R7+R_{BE})$, we have:
\vspace{-0.15cm}
\[
    R_{BF}=\dfrac{1}{\dfrac{1}{R_8} + \dfrac{1}{R_7 + R_{BE}}}=\dfrac{1}{\dfrac{1}{4} + \dfrac{1}{3+9}}=3(k\Omega)
\]
Finally, we calculate $R_{AF}$. Because $R1 + R_{BF}$, we have:
\vspace{-0.15cm}
\[
    R_{AF}=R_1 + R_{BF}=2 + 3=5 (k\Omega)
\]
By applying Ohm's law, we can find the voltage value between terminals A and F:
\vspace{-0.15cm}
\[
    V_{AF} = V = I \cdot R_{AF}=18 \cdot 5=90(V)
\]
We have voltages at nodes A, B, C, D, and E as follows:
\vspace{-0.15cm}
\[
    \begin{cases}
        V_{A}-V_{F}=V_{AF}=90 \Rightarrow V_{A}=90+V_{F} = 90+0=90(V)                      \\
        V_{BF} = I \cdot R_{BF}=18 \cdot 3=54(V) \Rightarrow V_{B}=V_{F}+V_{BF}=0+54=54(V) \\
    \end{cases}
\]
By applying the voltage divider rule, we have:
\vspace{-0.15cm}
\[
    V_{EF} = V_{BF} \cdot \dfrac{R7}{R_{BE}+R7}=54 \cdot \dfrac{9}{3+9}=40.5(V) \Rightarrow V_{E}=V_{F}+V_{EF}=0+40.5=40.5(V)                                                            \\
\]
\[
    \begin{aligned}
        V_{CE}      = & V_{BE} \cdot \dfrac{R_{CE}}{R_{CE}+R_{DE}}= (V_B-V_E)\cdot \dfrac{R_{CE}}{R_{CE}+R_{DE}}= (54-40.5) \cdot \dfrac{4}{4+2}=9(V) \\
        \Rightarrow   & V_{C}=V_{E}+V_{CE}=40.5+9=49.5(V)
    \end{aligned}
\]
\[
    \begin{aligned}
        V_{DE}      = & V_{CE} \cdot \dfrac{R_{DE}}{R_{DE}+R_{3}}= (V_C-V_E) \cdot \dfrac{R_{DE}}{R_{DE}+R_{3}}= (49.5-40.5) \cdot \dfrac{2}{2+10}=1.5(V) \\
        \Rightarrow   & V_{D}=V_{E}+V_{DE}=40.5+1.5=42(V)
    \end{aligned}
\]

\textbf{Conclusion:}
After rearranging the circuit and calculating step-by-step, we have:
\[
    \begin{cases}
        R_{AF}=5 (k\Omega)                                                  \\
        V_{A}=90(V), V_{B}=54(V), V_{C}=49.5(V), V_{D}=42(V), V_{E}=40.5(V) \\
    \end{cases}
\]

\subsubsection{Simulation}
To verify the calculation above, we did perform the simulation twice: first, for original circuit; second, for rearranged circuit. The results are as follows:
% display to image here on the same line with two caption
\begin{figure}[H]
    \centering
    \begin{subfigure}{0.45\textwidth}
        \centering
        \includegraphics[width=\linewidth]{graphics/lab1_ex3_sim.jpg}
        \caption{Simulation for original circuit}%
        \label{fig:6_3_sim1}
    \end{subfigure}
    \hfill
    \begin{subfigure}{0.45\textwidth}
        \centering
        \includegraphics[width=\linewidth]{graphics/lab1_ex3_sim_rear.jpg}
        \caption{Simulation for rearranged circuit}%
        \label{fig:6_3_sim2}
    \end{subfigure}
    \caption{Simulation results}%
    \label{fig:6_3_simulation_results}
\end{figure}
From the simulation results in Figure \ref{fig:6_3_simulation_results}, we can see that the equivalent resistance $R_{AF}$ and voltages at nodes A, B, C, D, and E are the same for both original and rearranged circuits. The simulation results confirm our calculations are correct.