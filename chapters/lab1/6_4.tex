\subsection{Exercise 4}
Given the following circuit, find $I_1$, $I_2$, $I_3$, $V_a$, and $V_b$. Present your calculation steps and check them out by performing the simulation.

\begin{figure}[H]
    \centering
    \includegraphics[width=0.7\textwidth]{graphics/lab1_ex4.jpg}
    \caption{Find $I_1$, $I_2$, $I_3$, $V_a$, and $V_b$}
    \label{fig:6_4_circuit}
\end{figure}

\subsubsection{Calculation}
The whole circuit equivalent resistance:
\vspace{-0.15cm}
\[
    R_{eq} = 9+\frac{1}{\frac{1}{6}+\frac{1}{3+3}} = 9+\frac{1}{\frac{1}{6}+\frac{1}{6}} = 12 (\Omega)
\]

By applying Ohm's law, we can find the total current $I_1$:
\vspace{-0.15cm}
\[
    I_1 = \frac{V}{R_{eq}} = \frac{12}{12} = 1 (mA)
\]

By the current division rule, we can find $I_2$ and $I_3$:
\vspace{-0.15cm}
\begin{align*}
    I_2 & = I_1 \times \frac{6}{6+3+3} = 1 \times \frac{6}{12} = 0.5 (mA)   \\
    I_3 & = I_1 \times \frac{3+3}{6+3+3} = 1 \times \frac{6}{12} = 0.5 (mA)
\end{align*}

By applying Ohm's law, we can find $V_a$ and $V_b$:
\vspace{-0.15cm}
\begin{align*}
    V_a & = I_2 \times 6 = 0.5 \times 6 = 3 (V)   \\
    V_b & = I_3 \times 3 = 0.5 \times 3 = 1.5 (V)
\end{align*}
\subsubsection{Simulation}
By performing the simulation in PSpice for TI, we have the following results:
\begin{figure}[H]
    \centering
    \includegraphics[width=0.6\textwidth]{graphics/lab1_ex4_sim.jpg}
    \caption{Simulation results of Exercise 4}%
    \label{fig:6_4_simulation}
\end{figure}
As shown in Figure \ref{fig:6_4_simulation}, the simulation results match our calculation.

