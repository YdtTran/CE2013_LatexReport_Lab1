\subsection{Zener Diodes as Regulators}
The Zener diode has a well-defined reverse-breakdown voltage, at which it starts con ducting current, and continues operating continuously in the reverse-bias mode without getting damaged. Additionally, the voltage drop across the diode remains constant over a wide range of voltages, a feature that makes Zener diodes suitable for use in voltage regulation.
\begin{figure}[H]
    \centering
    \includegraphics[width=0.6\textwidth]{graphics/lab2_8.jpg}
    \caption{Electrical characteristic of Zener diode\cite{lab_ref}}
    \label{fig:zener_regulator}
\end{figure}

In this exercise, a Zener diode is used to design a voltage regular circuit. The schematic in this exercise is given following Figure \ref{fig:zener_regulator}.
\begin{figure}[H]
    \centering
    \includegraphics[width=0.7\textwidth]{graphics/lab2_8_sche.jpg}
    \caption{Voltage regulator using Zener diode\cite{lab_ref}}
    \label{fig:zener_circuit}
\end{figure}
% \vspa
The Zener component in the circuit can be found in the Favourites list by searching the keyword Zener. The full name of the component used in the circuit above is \textbf{Zenner\_P - Zener Diode (parameterized)}. The default Zener voltage of this component is $V_Z = 5V$. However, this value can be changed in the properties of the component (right click and select Edit Properties) for other simulations.
Theory calculation
\begin{itemize}
    \item $I_L = \dfrac{V_Z}{R_L}$ (Ohm's Law).
    \item $I_S = \dfrac{V_{CC} - V_Z}{R_S}$ (KVL and Ohm's Law).
    \item $I_Z = I_S - I_L$ (KCL).
    \item $P_{RS} = I_S^2 \times R_S$.
    \item $P_{Z} = I_Z \times V_Z $.
\end{itemize}

Then, perform the calculation for the Zener diode voltage regulator with two different input voltage, including 8V and 12V power supply. Finally, run the simulations in PSpice (in Bias Point simulation profile) to confirm with the theory calculation.

The results are summarized in the table below.

\begin{table}[h!]
    \centering
    \renewcommand{\arraystretch}{1.3}
    \begin{tabular}{|c|c|c|c|c|c|c|c|c|c|c|c|c|}
        \hline
        \multirow{2}{*}{}                                &
        \multicolumn{6}{c|}{\textbf{Theory Calculation}} &
        \multicolumn{6}{c|}{\textbf{PSpice Simulation}}                                                                                                                      \\ \cline{2-13}
                                                         & $I_S$ & $I_L$   & $I_Z$    & $V_L$ & $P_{RS}$ & $P_Z$
                                                         & $I_S$ & $I_L$   & $I_Z$    & $V_L$ & $P_{RS}$ & $P_Z$                                                             \\ \hline
        Vcc = 8V                                         & $15$  & $3.33 $ & $11.67 $ & $5$   & $0.045 $ & $0.06 $ & $15.04 $ & $3.33 $ & $11.71 $ & $5$ & $0.05 $ & $0.06 $ \\ \hline
        Vcc = 12V                                        & $35$  & $3.33 $ & $31.67 $ & $5$   & $0.245 $ & $0.16 $ & $34.96 $ & $3.33 $ & $31.62 $ & $5$ & $0.25 $ & $0.16 $ \\ \hline
    \end{tabular}
    \caption{Comparison between theoretical and PSpice simulation results for Zener regulator circuit}
\end{table}
The following images show the PSpice simulation results for both input voltages.
% create two images side by side use subfigure
\begin{figure}[H]
    \centering
    \begin{subfigure}{0.45\textwidth}
        \centering
        \includegraphics[width=\textwidth]{graphics/lab2_8_8V.jpg}
        \caption{PSpice simulation result for Vcc = 8V}
        \label{fig:zener_8V}
    \end{subfigure}
    \hfill
    \begin{subfigure}{0.45\textwidth}
        \centering
        \includegraphics[width=\textwidth]{graphics/lab2_8_12V.jpg}
        \caption{PSpice simulation result for Vcc = 12V}
        \label{fig:zener_12V}
    \end{subfigure}
    \caption{PSpice simulation results for Zener regulator circuit at different input voltages}
\end{figure}
%  • IL= ...................................................................................
%  • IS= ...................................................................................
%  • IZ= ...................................................................................
%  • PRS=..................................................................................
%  • PZ=...................................................................................